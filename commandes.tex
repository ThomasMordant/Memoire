%%%%%%%%%%%%%%%%%%%%%%%%%%%%%%%%%%%%%%%%%%%%%%%%%%%%%%%%%
% --- Commandes communes aux classes personnalisées --- %
%%%%%%%%%%%%%%%%%%%%%%%%%%%%%%%%%%%%%%%%%%%%%%%%%%%%%%%%%

% --- Inclusion d'image ---

\newcommand{\inc}[1]{\includegraphics[width=\textwidth]{#1}}

% --- Pour des mathématiques en gras dans les titres ---

\DeclareRobustCommand*{\bfseries}{%
	\not@math@alphabet\bfseries\mathbf%
	\fontseries\bfdefault\boldmath\selectfont%
}

% --- Commandes d'spacement ---

\newcommand{\bigspace}{\par\addvspace{2\baselineskip}\par}

% --- Commandes de simplifications des mathématiques ---
%%%%%%%%%%%%%%%%%%%%%%%%%%%%%%%%%%%%%%%%%%%%%%%%%%%%%%%%

% --- Ensembles de nombres ---

\newcommand\N{\mathbf{N}}
\renewcommand\P{\mathbf{P}}
\newcommand\Z{\mathbf{Z}}
\newcommand\Q{\mathbf{Q}}
\newcommand\D{\mathbf{D}}
\newcommand\I{\mathbf{I}}
\newcommand\R{\mathbf{R}}
\newcommand\C{\mathbf{C}}
\newcommand\K{\mathbf{K}}
\newcommand\F{\mathbf{F}}
\newcommand{\iC}{\hat{\C}}
\newcommand{\bS}{\mathbf{S}}
\newcommand{\bF}{\mathbf{F}}

% --- Lettres majuscules calligraphiées ---

\newcommand{\cA}{\mathcal{A}}
\newcommand{\cB}{\mathcal{B}}
\newcommand{\cC}{\mathcal{C}}
\newcommand{\cD}{\mathcal{D}}
\newcommand{\cE}{\mathcal{E}}
\newcommand{\cF}{\mathcal{F}}
\newcommand{\cG}{\mathcal{G}}
\newcommand{\cH}{\mathcal{H}}
\newcommand{\cL}{\mathcal{L}}
\newcommand{\cM}{\mathcal{M}}
\newcommand{\cO}{\mathcal{O}}
\newcommand{\cP}{\mathcal{P}}
\newcommand{\cQ}{\mathcal{Q}}
\newcommand{\cR}{\mathcal{R}}
\newcommand{\cS}{\mathcal{S}}
\newcommand{\cU}{\mathcal{U}}
\newcommand{\cV}{\mathcal{V}}
\newcommand{\cX}{\mathcal{X}}
\newcommand{\cY}{\mathcal{Y}}
\newcommand{\cZ}{\mathcal{Z}}
\newcommand{\cJ}{\mathcal{J}}
\newcommand{\cW}{\mathcal{W}}

% --- Minuscules droites ---

\newcommand{\txtz}[1]{\,{\mathrm #1}}
\newcommand{\mA}{\txtz{a}}
\newcommand{\mB}{\txtz{b}}
\newcommand{\mC}{\txtz{c}}
\newcommand{\mD}{\txtz{d}}
\newcommand{\mE}{\txtz{e}}
\newcommand{\mF}{\txtz{f}}
\newcommand{\mG}{\txtz{g}}
\newcommand{\mH}{\txtz{h}}
\newcommand{\mI}{\txtz{i}}
\newcommand{\mJ}{\txtz{j}}
\newcommand{\mK}{\txtz{k}}
\newcommand{\mL}{\txtz{l}}
\newcommand{\mM}{\txtz{m}}
\newcommand{\mN}{\txtz{n}}
\newcommand{\mO}{\txtz{o}}
\newcommand{\mP}{\txtz{p}}
\newcommand{\mQ}{\txtz{q}}
\newcommand{\mR}{\txtz{r}}
\newcommand{\mS}{\txtz{s}}
\newcommand{\mT}{\txtz{t}}
\newcommand{\mU}{\txtz{u}}
\newcommand{\mV}{\txtz{v}}
\newcommand{\mW}{\txtz{w}}
\newcommand{\mX}{\txtz{x}}
\newcommand{\mY}{\txtz{y}}
\newcommand{\mZ}{\txtz{z}}

% --- Lettres majuscules droites ---

\newcommand{\rE}{{\rm E}}
\newcommand{\rT}{{\rm T}}
\newcommand{\rX}{{\rm X}}
\newcommand{\rY}{{\rm Y}}
\newcommand{\rO}{{\rm O}}
\newcommand{\oo}{{\rm o}}
\newcommand{\rL}{\mathrm{L}}

\newcommand{\matrice}[1]{\begin{pmatrix}#1\end{pmatrix}}
\newcommand{\systeme}[1]{\left\{\begin{aligned} #1 \end{aligned}\right.}
\newcommand{\vect}[1]{\overrightarrow{#1}}
\newcommand{\nens}[1]{\mathcal{#1}}

%---- Modifications de symboles -----

\newcommand{\ii}{\boldsymbol{i}}
\newcommand{\bs}[1]{\boldsymbol{#1}}
\newcommand{\jac}{{\rm J}}

\renewcommand{\epsilon}{\varepsilon}
\newcommand{\re}{\mathop{\mathrm{Re}}\nolimits}
\newcommand{\im}{\mathop{\mathrm{Im}}\nolimits}
\newcommand{\eps}{\varepsilon}
\newcommand{\ssi}{\iff}
\newcommand{\implique}{\implies}
\newcommand{\vers}{\rightarrow}
\newcommand{\donne}{\mapsto}
\newcommand{\m}{^{-1}}
\newcommand{\Sup}{{\mathrm{Sup}}\:}

% --- Délimiteurs ---

\newcommand{\Par}[1]{\left(#1\right)}
\newcommand{\parent}[1]{\left( #1 \right)}
\newcommand{\absolue}[1]{\left\lvert#1\right\rvert}
\newcommand{\abs}[1]{\absolue{#1}}
\newcommand{\norme}[1]{\left\lVert#1\right\rVert}
\newcommand{\ens}[1]{\left\{#1\right\}}
\newcommand{\ent}[1]{\left\lfloor #1 \right\rfloor}
\newcommand{\enstq}[2]{\ens{#1 \,\middle\vert\, #2}}
\newcommand{\barre}[1]{\overline{#1}}

% --- Nouveaux opérateurs ---

\DeclareMathOperator*{\Card}{Card}
\DeclareMathOperator*{\card}{card}
\DeclareMathOperator*{\pgcd}{pgcd}
\DeclareMathOperator*{\ppcm}{ppcm}
\DeclareMathOperator*{\val}{val}
\newcommand{\id}{\mathop{\mathrm{id}}\nolimits}
\DeclareMathOperator*{\tr}{tr}

\newcommand{\ch}{\mathop{\mathrm{ch}}\nolimits}
\newcommand{\sh}{\mathop{\mathrm{sh}}\nolimits}
\renewcommand{\th}{\mathop{\mathrm{th}}\nolimits}
\newcommand{\Arcsin}{\mathop{\mathrm{Arcsin}}\nolimits}
\newcommand{\Arccos}{\mathop{\mathrm{Arccos}}\nolimits}
\newcommand{\Arctan}{\mathop{\mathrm{Arctan}}\nolimits}
\newcommand{\Argsh}{\mathop{\mathrm{Argsh}}\nolimits}
\newcommand{\Argch}{\mathop{\mathrm{Argsh}}\nolimits}
\newcommand{\Argth}{\mathop{\mathrm{Argth}}\nolimits}
\renewcommand{\arcsin}{\Arcsin}
\renewcommand{\arccos}{\Arccos}
\renewcommand{\arctan}{\Arctan}
\newcommand{\argsh}{\Argsh}
\newcommand{\argch}{\Argch}
\newcommand{\argth}{\Argth}

\DeclareMathOperator*{\rg}{rg}
\DeclareMathOperator*{\Com}{Com}
\DeclareMathOperator*{\Fr}{Fr}
\DeclareMathOperator*{\diam}{diam}

\DeclareMathOperator{\Jac}{Jac}
\DeclareMathOperator{\GL}{GL}
\DeclareMathOperator{\Aut}{Aut}
\DeclareMathOperator{\SL}{SL}
\DeclareMathOperator{\PSL}{PSL}
\DeclareMathOperator{\PGL}{PGL}
\DeclareMathOperator{\Orth}{O}
\renewcommand{\det}{{\rm det}}

\newcommand*{\fonc}[5]{%
	#1 \colon \left|
	\begin{aligned}
		#2 &\to     #3 \\
		#4 &\mapsto #5
	\end{aligned}
	\right.\kern-\nulldelimiterspace
}

% --- Simplification des symboles ---

\newcommand{\transp}[1]{{^t}#1}
\newcommand{\relation}{\nens{R}}
\newcommand{\union}{\cup}
\newcommand{\Union}{\bigcup}
\newcommand{\inter}{\cap}
\newcommand{\Inter}{\bigcap}
\newcommand{\dans}{\subset}
\newcommand{\danseq}{\subseteq}
\newcommand{\contient}{\supset}
\newcommand{\contienteq}{\supseteq}
\newcommand{\disj}{\vee}

\DeclareMathOperator*{\Ker}{Ker}

\newcommand{\ega}{&=}
\newcommand{\rond}{\circ}
\newcommand{\fcaract}{\chi}
\newcommand{\prive}[1]{\setminus \ens{#1}}
\newcommand{\vide}{\emptyset}

\newcommand{\fois}{\times}
\newcommand{\congru}{\equiv}
\newcommand{\ou}{\text{ ou }}
\newcommand{\et}{\text{ et }}
\newcommand{\elem}[2]{\text{#1}\indice{#2}}
\newcommand{\indice}[1]{_\text{#1}}

% --- Éléments différentiels dans une intégrale ---

\newcommand{\dd}{\mathop{\mathrm{{}d}}\mathopen{}}
\newcommand{\dt}{\dd t}
\newcommand{\dx}{\dd x}
\newcommand{\du}{\dd u}
\newcommand{\diff}{\frac\dd\dt}
\newcommand{\diffat}[1]{\left.\diff\right\rvert_{t = #1}}

\renewcommand{\at}[2]{\left. #1\right\rvert_{#2}}

\newcommand{\point}[1]{\dot{#1}}
\newcommand{\dpoints}[1]{\ddot{#1}}

\newcommand{\vectoriel}{\wedge}
\newcommand{\grad}[1]{\vect{\nabla} \left( #1 \right)}

\newcommand{\inv}[1]{\overset{-1}{#1}}
\renewcommand{\nmid}{\diagup\hskip-11pt\mid}
\newcommand{\kpn}[2]{\binom{#2}{#1}}
\newcommand{\ind}[1]{#1\index{#1}}
\newcommand{\graphf}[3]{\sageplot[width=\textwidth]{plot(#1,(x,#2,#3))}}

\newcommand{\priv}{\setminus}
\newcommand{\rep}{(O,\vect{\imath},\vect{\jmath})}

\newcommand{\e}{\mathrm{e}}
\newcommand{\z}[2]{#1\e^{i\left(#2\right)}}

\newcommand{\cqfd}{\ifmmode\tag*{$\blacksquare$}\else\hfill$\blacksquare$\fi}
\newcommand{\fdef}{\begin{center}$\clubsuit$\end{center}}
\newcommand{\ouv}[1]{\overset{\circ}{#1}}

% --- Définitions des points ---

\newcounter{propiC}
\setcounter{propiC}{0}
\newcommand{\propt}[1]{\refstepcounter{propiC}\textbf{\thepropiC. \ }\label{#1}}
\newcommand{\newpropt}{\setcounter{propiC}{0}}

% --- Enoncés ---
%%%%%%%%%%%%%%%%%

% --- Définition de style de théorème ---

\newtheoremstyle{myplain}
	{\z@}
	{\z@}
	{}
	{}
	{\scshape}
	{}
	{ }
	{}

\patchcmd{\@begintheorem}{\ignorespaces}{\strut\par\nobreak\ignorespaces\setlist{nolistsep}}{}{}

\theoremstyle{myplain}

% --- Encadrement ---

\RequirePackage{mdframed}

\newcommand{\@expandNotes}{}
\newcounter{numberfootnote}
\newenvironment*{encadrer}[1][1pt]{%
	\Saut
	\setcounter{numberfootnote}{\number\value{footnote}}
	\renewcommand{\@expandNotes}{}
	\renewcommand{\note}[1]{%
		\footnotemark%
		\gappto{\@expandNotes}{%
			\addtocounter{numberfootnote}{1}%
			\footnotetext[\thenumberfootnote]{##1}%
		}
	}
	\begin{mdframed}[
		topline=false, bottomline=false, rightline=false, linewidth=#1,
		skipabove=0pt, skipbelow=0pt,
		innertopmargin=0pt, innerbottommargin=0pt,
		everyline=true, splittopskip=0pt, splitbottomskip=0pt
	]
}{%
	\end{mdframed}%
	\@expandNotes
}

% --- Définition des théorèmes ---

\newtheorem{definitionT}{Définition}[section]
\newtheorem{propositionT}[definitionT]{Proposition}
\newtheorem{theoremeT}[definitionT]{Théorème}
\newtheorem{loiT}[definitionT]{Loi}
\newtheorem*{demonstrationT}{Preuve}
\newtheorem*{preuveT}{Preuve}
\newtheorem{axiomeT}[definitionT]{Axiome}
\newtheorem{lemmeT}[definitionT]{Lemme}
\newtheorem{corollaireT}[definitionT]{Corollaire}
\newtheorem{procedeT}[definitionT]{Procédé}

% --- Théorèmes encadrés ---

\newcommand{\definition}[2]{\begin{encadrer}\begin{definitionT}[#2]#1\end{definitionT}\end{encadrer}}
\newcommand{\proposition}[2]{\begin{encadrer}\begin{propositionT}[#2]#1\end{propositionT}\end{encadrer}}
\newcommand{\theoreme}[2]{\begin{encadrer}\begin{theoremeT}[#2]#1\end{theoremeT}\end{encadrer}}
\newcommand{\loi}[2]{\begin{encadrer}\begin{loiT}[#2]#1\end{loiT}\end{encadrer}}
\newcommand{\demonstration}[2]{\begin{encadrer}[.3pt]\begin{demonstrationT}[#2]{\small#1}\end{demonstrationT}\end{encadrer}}
\newcommand{\preuve}[2]{\begin{encadrer}[.3pt]\begin{preuveT}[#2]{\small#1}\end{preuveT}\end{encadrer}}
\newcommand{\axiome}[2]{\begin{encadrer}\begin{axiomeT}[#2]#1\end{axiomeT}\end{encadrer}}
\newcommand{\lemme}[2]{\begin{encadrer}\begin{lemmeT}[#2]#1\end{lemmeT}\end{encadrer}}
\newcommand{\corollaire}[2]{\begin{encadrer}\begin{corollaireT}[#2]#1\end{corollaireT}\end{encadrer}}
\newcommand{\procede}[2]{\begin{encadrer}\begin{procedeT}[#2]#1\end{procedeT}\end{encadrer}}

%%%%%%%%%%%%%%%%%%%%%%%%%%%%%%%%%%%%%%%%%%%%
% --- Fin des commandes personnalisées --- %
%%%%%%%%%%%%%%%%%%%%%%%%%%%%%%%%%%%%%%%%%%%%
